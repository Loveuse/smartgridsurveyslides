% OSGP
% Attacking smart meters and smart devices
% False data injection
% Disconnect attack
% Jamming


\begin{frame}{Principali vulnerabilità delle Smart Grid: attacchi e contromisure}
	\begin{itemize}[<+- | alert@+>]
		\item Digitalizzazione delle infrastrutture critiche
		\begin{itemize}
			\item Aggiunta capacità computazione e comunicazione
		\end{itemize}
		\item Dispositivi non critici e disconnessi
		\begin{itemize}
			\item Connessi $\rightarrow$ Dati \textsc{process-critical}
		\end{itemize}
		\item Anello debole della catena
		\begin{itemize}
			\item Smart Meter
		\end{itemize}
		\item Nuova sfida per i produttori
	\end{itemize}
\end{frame}

%\begin{frame}{Principali vulnerabilità delle Smart Grid: attacchi e contromisure}
%	\begin{itemize}[<+- | alert@+>]
%		\item Security Testing
%		\item Open Smart Grid Protocol
%		\item False Data Injection
%		\item Disconnect Attack
%		\item Jamming
%	\end{itemize}
%\end{frame}

\plain{Security Testing}

\begin{frame}{Principali vulnerabilità delle Smart Grid: Security Testing}
		\textbf{Institute for Security and Open Methodologies (ISECOM)}
		\begin{block}{Open Source Security Testing Methodology Manual (OSSTMM)}
			\begin{itemize}
				\item Information Security
				\item Process Security
				\item \textbf{Internet Technology Security}
				\item Communications Security
				\item Wireless Security
				\item Physical Security
			\end{itemize}
			\end{block}
\end{frame}

\begin{frame}{Principali vulnerabilità delle Smart Grid: Security Testing}
\textbf{Internet Technology Security}
	\begin{itemize}[<+- | alert@+>]
		\item Kali Linux
		\item Wireshark
		\begin{itemize}
			\item Network Surveying
		\end{itemize}
		\item Nmap
		\begin{itemize}
			\item Port Scanning
			\item Services Identification
			\item System Identification
			\begin{itemize}
				\item Vulnerability Research and Verification
			\end{itemize}
			\item Denial of Service Testing
		\end{itemize}
		\item Nessus
		\begin{itemize}
			\item Internet Application Testing
		\end{itemize}
		\item Metasploit
		\begin{itemize}
			\item Exploit Testing
		\end{itemize}
	\end{itemize}
\end{frame}

\plain{Open Smart Grid Protocol}

\begin{frame}{Principali vulnerabilità delle Smart Grid: Open Smart Grid Protocol}
	\begin{itemize}[<+- | alert@+>]
		\item Comunicazione Smart meter $\leftrightarrow$ Aggregatore su Powerline
		\item Protocollo leggero
		\begin{itemize}
			\item Paga in sicurezza
		\end{itemize}
		\item Stream cipher RC4 - Ron Rivest, 1987 (mai pubblicato uff.)
		\begin{itemize}
			\item Periodicità nei primi 256 byte
			\item Forte correlazione fra chiave e keystream (\textit{Breaking 104 bit WEP in less than 60 seconds})
		\end{itemize}
		\item Funzione di digest lineare
		\begin{itemize}
			\item Ricomputazione del digest corretto senza il bisogno di cifratura
			\item Soggetta ad attacchi \textit{man-in-the-middle}
		\end{itemize}
		\item Secure Broadcast
		\begin{itemize}
			\item Sicurezza non nota
			\item Aggiornamenti firmware
		\end{itemize}
	\end{itemize}
\end{frame}

\plain{False Data Injection}

\begin{frame}{Principali vulnerabilità delle Smart Grid: False Data Injection}
	\begin{itemize}[<+- | alert@+>]
		\item Scambio informazioni stato
		\item Stima dello stato $\rightarrow$ Modello \textit{real-time}
		\item Manomettere le misure
		\begin{itemize}
			\item Frode
			\item Sovraccaricare l'infrastruttura
			\item Manipolare prezzi di mercato
		\end{itemize}
		\item \textit{Bad Data Injection}
		\begin{itemize}
			\item Rilevabile
		\end{itemize}
		\item \textit{Stealth Bad Data Injection}
		\begin{itemize}
			\item Non rilevabile
			\item Necessaria conoscenza della topologia
			\item Possibile inferire i parametri legati alla topologia
			\begin{itemize}
				\item \textit{Linear Independent Component Analysis}
			\end{itemize}
		\end{itemize}
	\end{itemize}
\end{frame}
% opzionale %%%%%%%%%%%%%%%%%%%%%%%%%%%%%%%%%%%%%%%%%%%%%%%%%%%%%%%%%%%%%%%%%%%%%
\begin{frame}{Principali vulnerabilità delle Smart Grid: attacchi e contromisure}
\textbf{Modello Matematico}
\begin{itemize}[<+- | alert@+>]
	\item Vettore delle misurazioni
	\begin{itemize}
		\item $\textbf{z} = \textbf{h(x)} + \textbf{e}$
		\begin{itemize}
			\item $\textbf{h(x)}$: relazione non lineare tra le misure $\textbf{z}$ e lo stato del sistema $\textbf{x}$
			\item $\textbf{e} = [e_1, \ldots, e_m]^T$, rumore Gaussiano delle misure
			\item Matrice di covarianza $\Sigma_e$
		\end{itemize}
	\item $\textbf{H} \in {\rm I\!R}$, definito come
	\begin{itemize}
		\item $\textbf{H}=\frac{\partial\textbf{h(x)}}{\partial\textbf{x}}|_{x=0}$
	\end{itemize}
	\end{itemize}
	\item Modello di approssimazione lineare della misura di corrente
	\begin{itemize}
		\item Misura sotto Operazioni Normali: $\textbf{z} =  \textbf{Hx} + \textbf{e}$
		\begin{itemize}
			\item Vettore di stato stimato: $\widehat{\textbf{x}} = (\textbf{H}^T\sum_e^{-1}\textbf{H})-1\textbf{H}^T\sum_e^{-1}\textbf{z}$
		\end{itemize}
	\end{itemize}
	\end{itemize}
\end{frame}

\begin{frame}{Principali vulnerabilità delle Smart Grid: attacchi e contromisure}
	\textbf{Modello Matematico}
	\begin{itemize}
	\item Bad Data Injection
	\begin{itemize}
		\item Misura sotto attacco non stealth: $\textbf{z}^\prime = \textbf{H}(\textbf{x}) + \textbf{b} + e$
		\item Vettore residuo: $\textbf{r} = \textbf{z} - \textbf{H}\widehat{\textbf{x}}$
		\item Rilevamento bad data: $max_i(|\textbf{r}_i|/\sqrt{cov(\textbf{r})}) \geq \gamma$
	\end{itemize}
	\item Stealth Bad Data Injection
	\begin{itemize}
		\item Misura sotto attacco stealth: $\textbf{z}^\prime = \textbf{H}(\textbf{x} + \delta\textbf{x}) + e$
	\end{itemize}
\end{itemize}
\end{frame}
%%%%%%%%%%%%%%%%%%%%%%%%%%%%%%%%%%%%%%%%%%%%%%%%%%%%%%%%%%%%%%%%%%%%%%%%%%%%%%%%%

\plain{Disconnect Attack}

\begin{frame}{Principali vulnerabilità delle Smart Grid: Disconnect Attack}
	\begin{itemize}[<+- | alert@+>]
		\item Connessione/Disconnessione Remota
		\item Attacco RCD $\rightarrow$ Blackout/Danni alla rete
		\item Difesa: ritardi casuali nell'esecuzione dei comandi RCD
		\begin{itemize}
			\item Prevenire rapidi cambiamenti del carico elettrico
			\item Tempo per rilevare e fermare un attacco in corso
		\end{itemize}
	\end{itemize}
\end{frame}

\plain{Jamming}

\begin{frame}{Principali vulnerabilità delle Smart Grid: Jamming}
	\begin{block}{}
	Strategia d'attacco utilizzata per la manipolazione del mercato elettrico
	\end{block}
	\pause
	\begin{block}{Assunzione}
	Si utilizza un sistema di comunicazione wireless, come \textbf{\color{blue_slides}WiMAX}, per effettuare il broadcast delle informazioni relative ai prezzi
	\end{block}
\end{frame}

\begin{frame}{Principali vulnerabilità delle Smart Grid: Jamming}
	\textbf{Attacco}
	\begin{enumerate}[<+- | alert@+>]
		\item L'attaccante fa Jamming in un'area molto popolata
		\item L'utente rimane a conoscenza del vecchio prezzo della corrente
		\item L'attaccante monitora il mercato elettrico
		\item Quando il prezzo cambia significativamente, si smette di fare Jamming
		\item Ogni utente adatta il proprio consumo energetico in base al nuovo prezzo
		%se il nuovo prezzo è superiore, l'utente diminuisce i consumi, facendo calare il prezzo con p alta. se il nuovo prezzo è più piccolo, l'utente incrementa i consumi, aumentando con alta probabilità il prezzo.
		\item L'attaccante può avere profitti da questa manipolazione del mercato
		%è in grado di predire come si comporteranno gli utenti alla ricezione del nuovo prezzo e quindi sono in grado di cambiare il prezzo della corrente quando e come vogliono.
	\end{enumerate}
\end{frame}

\begin{frame}{Principali vulnerabilità delle Smart Grid: Jamming}
	\begin{block}{Contromisure}
		Evitare di modificare il consumo di energia in maniera simultanea.\newline
		\textbf{\color{blue_slides}IDEA: } ci si basa su protocolli come Aloha e CSMA
	\end{block}
	\pause
	\begin{block}{}
	Si utilizza uno schema in cui un consumer sceglie un tempo casuale per cambiare la propria power response evitando che l'attaccante possa predire il comportamento dell'utente e quindi capire in che modo varia il prezzo della corrente
		%randomness: perchè i consumi sono in base alle esigenze e tendono a variare da persona a persona
		%utilizzando backoff casuali per evitare collisioni dovute a trasmissioni simultanee
		%schema di backoff casuale: un consumer sceglie un tempo casuale per cambiare la propria power response
		%si diffondono info relative ai cambiamenti in un intervallo di tempo piu ampio cosicchè la randomness del mercato possa contrattaccare i cambiamenti dei jammed user
	\end{block}
\end{frame}